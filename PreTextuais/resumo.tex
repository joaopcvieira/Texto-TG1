A proposta de exploração visa desenvolver um agente baseado em LLM que explora a automatização das três etapas centrais de uma Rede Bayesiana (BBN) para análise de riscos em projetos aeroespaciais: (i) elencar fatores relevantes, (ii) definir a topologia empregando métodos estruturados como DEMATEL/Delphi e (iii) estimar as probabilidades condicionais das CPTs — reduzindo o gargalo de tempo, custo e dependência de especialistas.

Os objetivos deste trabalho consistem em desenvolver uma BBN utilizando LLM, compará-la com uma BBN elaborada por especialistas, validar as discrepâncias junto a autoridades do setor e fornecer uma metodologia/agent AI que documente suas limitações e alcance. As questões de pesquisa visam confirmar se o agente identifica as variáveis essenciais, configura uma topologia adequada (por meio de DEMATEL ou Delphi), atribui probabilidades coerentes e obtém resultados semelhantes aos humanos. Em particular, o método DEMATEL fornece inicialmente os arcos de causa e efeito e as probabilidades de partida, que podem ser posteriormente refinados com dados ou métodos bayesianos.

Em síntese, a pesquisa pretende provar que a integração LLM + DEMATEL torna a construção e manutenção de BBNs mais ágil e escalável, permitindo iterar rapidamente cenários “what-if” e apoiar decisões críticas ao longo do ciclo de vida de sistemas aeroespaciais.
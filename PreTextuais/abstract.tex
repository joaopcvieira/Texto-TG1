The objective of the proposed investigation is to develop an agent based on large language models (LLMs) that studies the automation of the three principal stages of constructing a Bayesian Belief Network (BBN) for risk analysis in aerospace projects: (i) identification of relevant variables, (ii) determination of the network topology using structured methods such as DEMATEL/Delphi, and (iii) estimation of the conditional probability tables. This approach seeks to mitigate the constraints imposed by time, cost, and reliance on expert judgment.

The aims of this study include the development of a BBN utilizing LLM technology, comparative analysis against a BBN constructed by domain experts, validation of any discrepancies with industry professionals, and the provision of an AI-based agent that transparently documents its limitations and scope. The research questions evaluate whether the agent successfully identifies essential parameters, configures an appropriate topology (employing DEMATEL or Delphi methodologies), assigns cogent probabilities, and yields outcomes comparable to those produced by human experts. Notably, the DEMATEL method initially establishes the cause-and-effect structure and preliminary probability estimates, which may subsequently be refined through data-driven or Bayesian techniques.

In conclusion, the research endeavor aims to demonstrate that the integration of LLMs with the DEMATEL approach enhances the agility and scalability of BBN construction and maintenance, thereby facilitating rapid iteration of "what-if" scenarios and underpinning critical decision-making processes throughout the lifecycle of aerospace systems.